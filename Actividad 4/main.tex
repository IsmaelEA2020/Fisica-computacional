\documentclass[12pt]{article}
\usepackage[a4paper]{geometry}
%\userpackage[top=1 in, bottom=1.25 in, left=1.1 in, rigth=1.1 in] {geometry}
%\usepackage[paperwidth=17cm, paperheight=22.5cm, bottom=2.5cm, right=2.5cm]{geometry}
\usepackage[utf8]{inputenc}
%\usepackage[a4paper, top=2.5cm, bottom=2.5cm, left=2.2cm, right=2.2cm]
%{geometry}
%\usepackage[myheadings]{fullpage}
\usepackage{fancyhdr}
\usepackage{lastpage}
%\usepackage{float}
\usepackage{graphicx, wrapfig, subcaption, setspace, booktabs}
\usepackage{graphicx}
\usepackage[T1]{fontenc}
\usepackage[font=small, labelfont=bf]{caption}
%\usepackage{fourier}
\usepackage[protrusion=true, expansion=true]{microtype}
\usepackage[english]{babel}
\usepackage{sectsty}
\usepackage{url, lipsum}
\usepackage[T1]{fontenc}
\usepackage{icomma}
\usepackage{siunitx}
\usepackage{ragged2e}
\usepackage{amsmath}
\usepackage{comment}
\usepackage{enumerate}
%\usepackage{changepage}
\usepackage{anysize}




\newcommand{\HRule}[1]{\rule{\linewidth}{#1}}
\onehalfspacing
\setcounter{tocdepth}{5}
\setcounter{secnumdepth}{5}

%-------------------------------------------------------------------------------
% HEADER & FOOTER
%-------------------------------------------------------------------------------


\begin{comment}
-Udledninger
$$
\begin{aligned}
\end{aligned}
$$

-Opgavetekst
\begin{figure}[H]
\includegraphics[width=0.5\textwidth]{"path"}
\end{figure} 


-Opgave billede med tekst
\begin{figure}[H]
\caption{"Billedtekst"}
\includegraphics[width=0.5\textwidth]{"path"}
\end{figure} 

-Værdier
$\\
$


\end{comment}
\begin{document}

\begin{titlepage}

\title{ \normalsize 
		%\begin{figure}
        \begin{center}
        \includegraphics[height=6cm]{Logo.jpg}
        \end{center}
       % \end{figure}
        \LARGE \textsc{\textbf{Universidad De Sonora}} \\ \bigskip
		\Large División de Ciencias Exactas y Naturales \\
        Licenciatura En Física \\ \bigskip
        \bigskip
        Física Computacional I
		\\ [0.1cm]  
		\HRule{2pt} \\
		\Large \textbf{{Actividad 4}} \\
        \textit{\textbf{"Análisis Exploratorio de Datos en Python"}}
		\HRule{2pt} \\
		\normalsize \vspace*{0.001\baselineskip}}
        
\date{\bigskip \Large  \hspace*{\fill} Hermosillo, Sonora a febrero 05 de 2021}

        
\author{
		\Large\textbf{ Ismael Espinoza Arias} \\ \bigskip
        \\ \bigskip
       \Large Profesor Carlos Lizárraga Celaya}
       \end{titlepage}
       \maketitle
       

\newpage
\pagestyle{plain}
\section{Introducción}
 En esta actividad iniciamos en la exploración de datos en Python, vimos las distintas funciones que tienen para graficar las nuevas librerías utilizadas. También vimos como es que se manejan los datos con estos tipos de funciones.


\section{Desarrollo}
Para la realización de esta actividad, fue necesaria la asesoría del profesor y revisar el contenido abiertamente, para poder llegar a la utilización de los datos de la manera que se esperaba, ya que como pudimos ver que los datos que manejamos ya tienen incluidos sus gráficas de con respecto a sus datos, y gracias a ello pudimos darnos una idea de como es que las nuestras iban a salir o como es que se debían de comportar al momento de ser del mismo estilo.

\section{Conclusión}
Como conclusión, podemos decir que en esta actividad se vio profundamente el uso de nuevas librerías que nos ayudaron a hacer unas mejores gráficas, o al menos graficas alternativas que nos ayudan a interpretar o manejar los datos de diferentes maneras según sea el caso, es por eso que me gustó mucho esta actividad porque se relaciona con la ciencia de datos, rama que en un futuro pienso enfocarme para poder concluir mis estudios en un futuro.


\section{Apéndice}
\begin{enumerate}
\item \textbf{¿Qué te pareció?}\\
\textit{Me pareció muy bien la actividad, ya que como vimos como manejar los datos con Python y de forma introductoria alas gráficas de otro estilo en Python.}

\item \textbf{¿Cómo estuvo el reto?}\\
\textit{Estuvo algo difícil, ya que no pude compilar en ciertas ocasiones y fue muy difícil encontrar el error, peor al final todo salió bien y me gusto la actividad.}

\item \textbf{¿ ¿Qué se te dificultó más??} \\
\textit{El aprender a usar la nueva librería , ya que nunca antes lo habia intentado usar, pero lo bueno es que tengo compañeros que si saben y que me ayudaorn bastante.}

\item \textbf{¿Qué te aburrió?}\\
\textit{Nada en concreto, todo creo que fue de suma importancia y que en todo momento todo fue muy activo y dinámico.}

\item \textbf{¿Qué recomendarías para mejorar la primera Actividad?? }\\
\textit{Que también este el tiempo del sábado y del domingo para realizar la actividad, ya que me gustaría tener el fin de semana extra por si se necesitaban esos días extras.}

\item \textbf{¿Que grado de complejidad le asignarías a esta Actividad? (Bajo, Intermedio, Avanzado)} \\
\textit{Le asignaría un mediano, porque todavía no consideró que la actividad sea demasiado difícil como para que se vuelva intensivo el ejercicio..}

\end{enumerate}

\end{document}