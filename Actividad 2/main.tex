\documentclass[12pt]{article}
\usepackage[a4paper]{geometry}
%\userpackage[top=1 in, bottom=1.25 in, left=1.1 in, rigth=1.1 in] {geometry}
%\usepackage[paperwidth=17cm, paperheight=22.5cm, bottom=2.5cm, right=2.5cm]{geometry}
\usepackage[utf8]{inputenc}
%\usepackage[a4paper, top=2.5cm, bottom=2.5cm, left=2.2cm, right=2.2cm]
%{geometry}
%\usepackage[myheadings]{fullpage}
\usepackage{fancyhdr}
\usepackage{lastpage}
%\usepackage{float}
\usepackage{graphicx, wrapfig, subcaption, setspace, booktabs}
\usepackage{graphicx}
\usepackage[T1]{fontenc}
\usepackage[font=small, labelfont=bf]{caption}
%\usepackage{fourier}
\usepackage[protrusion=true, expansion=true]{microtype}
\usepackage[english]{babel}
\usepackage{sectsty}
\usepackage{url, lipsum}
\usepackage[T1]{fontenc}
\usepackage{icomma}
\usepackage{siunitx}
\usepackage{ragged2e}
\usepackage{amsmath}
\usepackage{comment}
\usepackage{enumerate}
%\usepackage{changepage}
\usepackage{anysize}




\newcommand{\HRule}[1]{\rule{\linewidth}{#1}}
\onehalfspacing
\setcounter{tocdepth}{5}
\setcounter{secnumdepth}{5}

%-------------------------------------------------------------------------------
% HEADER & FOOTER
%-------------------------------------------------------------------------------


\begin{comment}
-Udledninger
$$
\begin{aligned}
\end{aligned}
$$

-Opgavetekst
\begin{figure}[H]
\includegraphics[width=0.5\textwidth]{"path"}
\end{figure} 


-Opgave billede med tekst
\begin{figure}[H]
\caption{"Billedtekst"}
\includegraphics[width=0.5\textwidth]{"path"}
\end{figure} 

-Værdier
$\\
$


\end{comment}
\begin{document}

\begin{titlepage}

\title{ \normalsize 
		%\begin{figure}
        \begin{center}
        \includegraphics[height=6cm]{Logo.jpg}
        \end{center}
       % \end{figure}
        \LARGE \textsc{\textbf{Universidad de Sonora}} \\ \bigskip
		\Large División de Ciencias Exactas y Naturales \\
        Licenciatura en física \\ \bigskip
        \bigskip
        Física computacional I
		\\ [0.1cm]  
		\HRule{2pt} \\
		\Large \textbf{{Actividad 2}} \\
        \textit{\textbf{"Introducción a la programación en python"}}
		\HRule{2pt} \\
		\normalsize \vspace*{0.001\baselineskip}}
        
\date{\bigskip \Large  \hspace*{\fill} Hermosillo, Sonora a enero 23 de 2021}

        
\author{
		\Large\textbf{ Ismael Espinoza Arias} \\ \bigskip
        \\ \bigskip
       \Large Profesor Carlos Lizárraga Celaya}
       \end{titlepage}
       \maketitle
       

\newpage
\pagestyle{plain}
\section{Introducción}
 Python es un lenguaje interpretado de código abierto, multiparadigma y multiplataforma. Tiene sus inicios en 1991 con su autor Guido van Rossum y actualmente ocupa el tercer lugar en la popularidad de lenguajes en la comunidad de acuerdo al índice TIOBE. Cabe destacar que es un lenguaje de propósito general muy poderoso y flexible, a la vez que sencillo y fácil de aprender. Ahora la forma en la que aplicamos python para esta ocasión, fue que debíamos calcular el área y volumen ciertas figuras, comenzando con el circulo, después ir incrementando el nivel con una figura mas exótica, en este caso es la elipse; en este punto llegamos a los volúmenes de una esfera y de un cilindro. En el segundo apartado llegamos a construir un código capaz de resolver una ecuación de segundo grado excluyendo las raíces complejas, solo reales. En el ejercicio 3 programamos un script con la capacidad de calculare la raíz de cualquier numero que ingresemos utilizando ciertos métodos peculiares, y por último, el ejercicio que nos ayudo a dominar las gráficas, fue el de aproximar series de Taylor a un valor y observar las gráficas, en este caso se aproximo a la función ln(x+1). 

\section{Impresiones y experiencias}
	Mis primeras impresiones, fueron que realicé las mismas actividades que ya había hecho posteriormente en otro curso de programación, que en este caso es Fortran, así que no tuve muchas complicaciones para asimilar lo que me pedían y ver por donde atacar el problema. Aparte también pensé que la actividad iba a ser un poco mas difícil, cuando en realidad con el paso de la práctica fui investigando y solucionando los problemas que se me iban presentando, entonces se volvieron mas sencillos y tengo mas conocimientos para futuros trabajos.
	
\section{Bibliotecas}
Antes de iniciar con cualquier tipo de código es de suma importancia y recomendable agregar las bibliotecas para que se carguen a la memoria de las celdas. Se agregaron 3 muy importantes: Pandas, Matplotlib.pyplot y Numpy. Siendo Pandas usada para la manipulación en el manejo de datos y análisi para la programación en Phyton, Numpy un paquete indispensable que permite utilizar matrices, transformadas e inclusive códigos externos de otros lenguajes de programación y por último Matplotlib.pyplot utilizada para la realización de gráficas en 2 dimensiones para Phyton.\\


\section{Apéndice}
\begin{enumerate}
\item \textbf{¿Qué te pareció?}\\
\textit{Me pareció una actividad comprometedora ya que se veía que iba terminar con una gran experiencia al final de estas actividades.}

\item \textbf{¿Cómo estuvo estuvo la carga de trabajo?} \\
\textit{Estuvo algo pesada, pero el docente siempre estuvo al pendiente cualquier duda y todo el trabajo salió a consideración.}

\item \textbf{¿Qué se te dificultó más?}\\
\textit{El hecho de haber programado en pyhton sin haberlo tocado anteriormente, creo que fue lo que se me dificulto además que en una parte del proyecto que no se realizaba como yo tenía pensado, pero igual se resolvió de una u otra forma.}

\item \textbf{¿Qué te aburrió?}\\
\textit{Nada en concreto, todo creo que fue de suma importancia y que en todo momento todo fue muy activo y dinámico.}

\item \textbf{¿Qué recomendarías para mejorar la primera Actividad?? }\\
\textit{Que la actividad se fuera abriendo desde el principio de la semana, así el mismo día podría el alumno tratar de ingresar a ella y así tendrían mas días para dudas y consejos del profesor.}

\item \textbf{¿Qué grado de complejidad le asignarías a esta Actividad? (¿Bajo, Intermedio, ¿Avanzado?} \\
\textit{Al principio, la mire avanzada pero con el paso de la misma me fui inclinando a intermedio, es por eso que yo diría que hasta el momento la dificultad es media.}

\end{enumerate}

\end{document}